\documentclass[a4paper, twoside]{article}
\usepackage[utf8]{inputenc} % Especifica la codificación de caracteres de los documentos.
\usepackage[spanish]{babel} % Indica que el documento se escribirá en español.
\usepackage[top=3cm, bottom=2.5cm, inner=1.5cm, outer=2.5cm]{geometry} % Márgenes personalizados
\usepackage{subfiles} % Paquete para incluir el preambulo en los sub archivos.
\usepackage{afterpage} % Permite añadir páginas despues de una página dada.
\usepackage{hyperref} % Permite incluir enlaces en los archivos.
\usepackage{lastpage} % Paquete para poder contabilizar el total de páginas del documento.
\usepackage{fancyhdr} % Permite personalizar los header y footer del documento.
\usepackage{graphicx} % Permite incluir gráficos
\usepackage[hang, bf]{caption} % Personaliza los subtítulos de las figuras y tablas
\usepackage{float} % Permite posicionar mejor las figuras y tablas
\usepackage{amsmath} % Comandos para la escritura de fórmulas matemáticas de mayor complejidad
\usepackage{amsfonts} % Proporciona fuentes matemáticas
\usepackage{amssymb} % Proporciona símbolos matemáticos de la American Mathematical Society

% Defino la ruta de los paquetes personalizados para el apunte
\newcommand{\rutapaquetes}{./paquetes-apunte}

\usepackage[mostrarlicencia]{\rutapaquetes/caratula} % Caratula personalizada (cargada desde caratula.sty)
\usepackage{\rutapaquetes/macros} % Macros útiles para los apuntes (cargado desde macros.sty)
\usepackage[ocultarrevisores]{\rutapaquetes/colaboradores} % Seccion de colaboradores (cargada y creada con colaboradores.sty)
\usepackage{\rutapaquetes/historial} % Seccion de historial de cambios (cargada y creada con historial.sty)

% Define los estilos de los enlaces interpretados por el paquete hyperref
\hypersetup{
    colorlinks=true,   % false: boxed links; true: colored links
    linkcolor=black,   % color of internal links (change box color with linkbordercolor)
    citecolor=green,   % color of links to bibliography
    filecolor=magenta, % color of file links
    urlcolor=blue,     % color of external links
}

% Define los directorios de las imágenes y gráficos
\graphicspath{ {./} {\rutapaquetes/} }

\newcommand{\nombremateria}{Sistemas Gráficos (66.71 - 86.43)} % Defino el comando "\nombremateria" para no harcodear el nombre en varios lugares.

% Define el pagestyle personalizado
\pagestyle{fancy}
\fancyhf{}
\renewcommand{\sectionmark}[1]{\markboth{}{\thesection\ \ #1}}
% Define header para pagina par
\fancyhead[ER]{\rightmark}
% Define header para pagina impar
\fancyhead[OL]{\rightmark}
% Define footer para pagina par
\fancyfoot[EL]{\nombremateria} % Nombre del apunte a la izquierda
\fancyfoot[ER]{Página \thepage\ de \pageref{LastPage}} % Numero de pagina a la derecha
% Define footer para pagina impar
\fancyfoot[OL]{Página \thepage\ de \pageref{LastPage}} % Numero de pagina a la izquierda
\fancyfoot[OR]{\nombremateria} % Nombre del apunte a la derecha

\renewcommand{\footrulewidth}{0.4pt} % Agrego linea que separa el footer

% Configura la caratula
\materia{\nombremateria}
\tipoapunte{Teórico}
%\tema{Tema de la Materia}
%\subtema{Subtema}

\cuadro{definicion}{gray}
\cuadro{teorema}{blue}
\cuadro{corolario}{green}

\begin{document}
% Página en blanco agregada después de la carátula
%\afterpage{
%	\null
%	\thispagestyle{empty}%
%	\addtocounter{page}{-1}%
%	\newpage}
\maketitle % Genera la carátula

\tableofcontents % Genera el índice

\subfile{\rutapaquetes/acerca-del-proyecto.tex} % Inlcuye informacion acerca del proyecto FIUBA Apuntes

\section{3D}

\begin{definicion}{Sistema de coordenadas}
	Muchas plataformas gráficas usan diferentes sistemas de coordenadas. En la materia se utilizarán los sistemas de coordenadas de terna derecha.
\end{definicion}

\subsection{Manipulación de vértices}

\begin{definicion}{Orden de operaciones}
	Para una correcta manipulación de los vértices, se recomienda el siguiente ordeN de operaciones.
	\[
		ESCALADO \Rightarrow ROTACION \Rightarrow TRASLACION
	\]
\end{definicion}

\subsubsection{Escalado}
\begin{minipage}{0.3\textwidth}
    \begin{center}
		\begin{definicion}{Operación de escalado}
			\begin{align*}
				x' = e_x \cdot x \\
				y' = e_y \cdot y \\
				z' = e_z \cdot z \\
			\end{align*}
		\end{definicion}
    \end{center}
\end{minipage}
\begin{minipage}{0.6\textwidth}
    \begin{center}
		\begin{definicion}{Tipos de escalado}
			\begin{description}
				\item[Isotrópico] Cuando se cumple $ e_x = e_y = e_z $.
				\item[Anisotrópico] Cuando no se cumple la igualdad del isotrópico.
			\end{description}
		\end{definicion}
    \end{center}
\end{minipage}

\begin{definicion}{Proyección}
	Es un tipo particular de escalado donde 2 de sus índices tienen valor 1 y uno de ellos tiene valor 0.

	Como resultado, se obtendrá la proyección de los vértices en el plano correspondiente a los índices cuyo valor es 1.
\end{definicion}

\begin{definicion}{Escalado inverso}
	Permite regresar al tamaño original.

	No es válido para proyecciones, ya que se eliminó la información correspondiente a uno de los ejes.
\end{definicion}

\section{Sección de ejemplo}

Lorem ipsum dolor sit amet, consectetur adipiscing elit. Sed accumsan eros erat, bibendum porttitor elit gravida et. Phasellus vulputate scelerisque eros ut ultrices. Ut suscipit, justo ac viverra commodo, turpis massa lacinia nisl, ac commodo arcu sapien ut quam. Vestibulum sodales, erat nec molestie sagittis, orci eros posuere velit, ut sagittis purus nunc nec elit. Quisque in dolor eget quam tempor consectetur. Nam laoreet, enim venenatis facilisis pulvinar, leo tortor posuere tellus, pellentesque elementum metus risus a ex. Quisque semper id elit eu luctus. Duis commodo tincidunt vehicula. Praesent varius vel sapien semper facilisis. Integer sollicitudin urna lorem, mattis sagittis mauris varius vitae. Aenean cursus auctor ligula, in porta felis porttitor mattis. Sed et porta tellus. Donec et turpis vel erat interdum malesuada at ac elit. Aenean sed lorem venenatis, laoreet nibh vitae, sagittis massa. Etiam vel iaculis ex. Maecenas tincidunt erat eu odio tempus facilisis.

\begin{figure}[H] % Opción H coloca la figura en la posición en que se encuentra el código
	\centering % Centra la imagen horizontalmente
	\includegraphics[width=0.7\textwidth]{fiuba_apuntes} % Especifica el archivo y tamaño
	\caption{Logo de FIUBA Apuntes} % Agrega subtitulo a la imagen
	\label{fig:fiuba_apuntes} % Etiqueta para poder referenciar la figura
\end{figure}

In a condimentum massa, in volutpat purus. Phasellus pellentesque purus faucibus nibh tincidunt, commodo tincidunt dolor semper. Phasellus eget tincidunt elit. Aenean eu quam diam. Donec vel leo nisl. Mauris quis aliquam nulla. Proin ornare malesuada enim, sed maximus ligula congue a. Lorem ipsum dolor sit amet, consectetur adipiscing elit. Pellentesque habitant morbi tristique senectus et netus et malesuada fames ac turpis egestas. Cras dapibus lorem eu orci dignissim sollicitudin.

\begin{corolario*}{Titulo del corolario} % El * es para que no enumere el cuadro de corolario (no implementado aún)
	Suspendisse potenti. In at mauris in dolor interdum sodales. Nulla eget eros ipsum. Mauris aliquet turpis eu mollis elementum. Class aptent taciti sociosqu ad litora torquent per conubia nostra, per inceptos himenaeos. Donec dui risus, tristique id ipsum eu, consequat suscipit turpis. Donec sagittis aliquet neque ut malesuada. Mauris nec molestie tellus. Fusce quis scelerisque quam.
	
	\begin{equation*} % El * es para que no enumere la ecuación
		a^2 + b^2 = c^2
	\end{equation*}
\end{corolario*}

Vivamus congue interdum tortor. Duis facilisis sapien quis dui venenatis euismod. Mauris tristique, metus id imperdiet sodales, risus ex porta dolor, eget egestas arcu eros sed dolor. Aliquam erat volutpat. Sed eu justo justo. Quisque iaculis vestibulum urna sed fringilla. Nullam mattis vulputate tortor, at dictum metus finibus in. In luctus elit nec sem luctus imperdiet.

Pellentesque habitant morbi tristique senectus et netus et malesuada fames ac turpis egestas. Mauris tempor ipsum in dui pretium, eget sollicitudin est sagittis. Nullam viverra vulputate lorem, eu porttitor augue dictum rhoncus. Nullam risus nibh, rutrum at enim eu, rutrum malesuada metus. Sed pretium, orci non consectetur vulputate, orci magna pellentesque sem, ac aliquet sapien neque quis diam. Phasellus et dolor a nunc dapibus blandit. Suspendisse sit amet aliquet nibh, molestie mattis sapien. Cras sit amet sagittis ex. Donec nec sapien mollis, hendrerit neque sed, imperdiet augue. Sed dignissim eu eros nec aliquet. Ut convallis rutrum nunc at lacinia. 


% Bibliografía utilizada en el apunte
\newpage
\newcommand{\bibliographyname}{Bibliografía} % Defino el nombre de la sección de la bibliografía
\addcontentsline{toc}{section}{\bibliographyname} % Agrego la bibliografía en el índice
\renewcommand\refname{\bibliographyname} % Renombro a la bibliografía (por default es 'Referencias')
\begin{thebibliography}{X}
	\bibitem{Hea} \textsc{Donald Hearn} y \textsc{M. Pauline Baker}, \textit{Gráficos por computadora en OpenGL}, Tercera edición, Pearson.
	\bibitem{Dan} \textsc{David Wolff}, \textit{OpenGL 4.0 Shading Language Cookbook}.
\end{thebibliography}

% Incluir los nombres de las personas que han colaborado en la creación del apunte
\colaborador{Ezequiel Pérez Dittler}
%\colaborador{Colaborador 2}
%\revisor{Dr. Profesor}{10/01/2015}
\makeseccioncolaboradores % Crea la seccion de colaboradres

% Incluir el historial de cambios
\revision{25/08/2015}{Versión inicial.}
%\revision{10/01/2015}{Se agregó una carátula personalizada, una sección con información del proyecto FIUBA Apuntes, la sección de colaboradores del apunte, sección de historial de cambios, ejemplo de bibliografía, ajuste de márgenes, encabezado y pie de página.}
\makehistorial

\end{document}